% Options for packages loaded elsewhere
\PassOptionsToPackage{unicode}{hyperref}
\PassOptionsToPackage{hyphens}{url}
%
\documentclass[
]{article}
\usepackage{amsmath,amssymb}
\usepackage{lmodern}
\usepackage{iftex}
\ifPDFTeX
  \usepackage[T1]{fontenc}
  \usepackage[utf8]{inputenc}
  \usepackage{textcomp} % provide euro and other symbols
\else % if luatex or xetex
  \usepackage{unicode-math}
  \defaultfontfeatures{Scale=MatchLowercase}
  \defaultfontfeatures[\rmfamily]{Ligatures=TeX,Scale=1}
\fi
% Use upquote if available, for straight quotes in verbatim environments
\IfFileExists{upquote.sty}{\usepackage{upquote}}{}
\IfFileExists{microtype.sty}{% use microtype if available
  \usepackage[]{microtype}
  \UseMicrotypeSet[protrusion]{basicmath} % disable protrusion for tt fonts
}{}
\makeatletter
\@ifundefined{KOMAClassName}{% if non-KOMA class
  \IfFileExists{parskip.sty}{%
    \usepackage{parskip}
  }{% else
    \setlength{\parindent}{0pt}
    \setlength{\parskip}{6pt plus 2pt minus 1pt}}
}{% if KOMA class
  \KOMAoptions{parskip=half}}
\makeatother
\usepackage{xcolor}
\usepackage[margin=1in]{geometry}
\usepackage{graphicx}
\makeatletter
\def\maxwidth{\ifdim\Gin@nat@width>\linewidth\linewidth\else\Gin@nat@width\fi}
\def\maxheight{\ifdim\Gin@nat@height>\textheight\textheight\else\Gin@nat@height\fi}
\makeatother
% Scale images if necessary, so that they will not overflow the page
% margins by default, and it is still possible to overwrite the defaults
% using explicit options in \includegraphics[width, height, ...]{}
\setkeys{Gin}{width=\maxwidth,height=\maxheight,keepaspectratio}
% Set default figure placement to htbp
\makeatletter
\def\fps@figure{htbp}
\makeatother
\setlength{\emergencystretch}{3em} % prevent overfull lines
\providecommand{\tightlist}{%
  \setlength{\itemsep}{0pt}\setlength{\parskip}{0pt}}
\setcounter{secnumdepth}{-\maxdimen} % remove section numbering
\ifLuaTeX
  \usepackage{selnolig}  % disable illegal ligatures
\fi
\IfFileExists{bookmark.sty}{\usepackage{bookmark}}{\usepackage{hyperref}}
\IfFileExists{xurl.sty}{\usepackage{xurl}}{} % add URL line breaks if available
\urlstyle{same} % disable monospaced font for URLs
\hypersetup{
  pdftitle={Blog Post 7},
  pdfauthor={Daniel Sussman},
  hidelinks,
  pdfcreator={LaTeX via pandoc}}

\title{Blog Post 7}
\author{Daniel Sussman}
\date{2022-12-02}

\begin{document}
\maketitle

{
\setcounter{tocdepth}{3}
\tableofcontents
}
\t To continue our analysis, our team has decided to refine our models
for ease of explanation/visualization and summarize our data to join
with a secondary dataset. At the start of our project, we wanted to
assess health trends over the course of the past two decades and check
for health inequalities among races/ethnic groups. To accomplish this,
we created two separate multiple linear regression models for systolic
(SBP) and diastolic blood pressure (DBP). While this method was
serviceable, working with two separate outcomes made our data difficult
to visualize and model concisely. Thus, we've decided to change our
outcome variable to mean arterial pressure (MAP), which is essentially a
weighted average of the two values for blood pressure calculated with
the formula MAP =DBP+13(SBP-DBP) . Using this singular outcome allows us
to convey the same information in a more compact manner. We also plan on
adjusting this statistical model to treat ``white'' as the baseline for
race/ethnicity to assess whether other races differ in health outcomes
from white individuals.\n

\t To further explore our regression model, we must visualize the
coefficients of our dataset. We plan to create faceted plots with MAP on
the y-axis and a single predictor on the x-axis. All other predictors
will be held at a constant value. We will also bin the age variable to
more easily work with the data, separating the variable into 18--39,
40--59, and 60 and over similar to a CDC data brief
(\url{https://www.cdc.gov/nchs/products/databriefs/db289.htm\#}:\textasciitilde:text=The\%20prevalence\%20of\%20hypertension\%20increased,those\%20aged\%2060\%20and\%20over).
We also plan on expanding our model through the addition of a secondary
dataset. We have included information related to the percent of
government spending on general health per year and plan to visualize and
discuss our plots. Statistical analyses may not be feasible due to the
small number of observations in our appended dataset (10 rows), so we
will focus on creating visualizations. \n

\t The results of our multiple linear regression have verified our
initial hypothesis, so we plan on taking a deeper dive into this idea to
flesh out and provide evidence for our thesis. Specifically, we claim
that the outcome of blood pressure (MAP) is significantly associated
with race/ethnicity and the year of data collection when corrected for
other variables (age and gender). We plan to assess how population blood
pressure levels have changed over time and analyze the effects of
certain races/ethnic groups on MAP.

\end{document}
